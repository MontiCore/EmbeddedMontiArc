\documentclass[12pt, a4paper]{scrartcl}
\usepackage{inputenc}
\usepackage{amsmath}
\usepackage{amssymb}
\usepackage{amsthm}
\usepackage{dsfont}
\usepackage{apacite}

\begin{document}

\pagestyle{empty}

\section*{Proposal for Bachelorthesis}
\subsubsection*{Julian Dierkes}

Goal of this Bachelorthesis is to extend EmbeddedMontiArcDL (EMADL) with the special neural net
architecture of generative adversiarial networks (GANs). In this proposal we will shortly introduce
the already existing tools MontiCore, EmbeddedMontiArch and EmbeddedMontiArcDL. Finally we will
talk about GANs and their challenges when integrating them to EMADL.

\bigskip

MontiCore is a language workbench which is capable of generating generators for domain specific
languages (DSL). When fed with a context free grammar of a DSL and additional parameters MontiCore 
generates tools like a language parser and components for an abstract syntax tree. These tools
can then be used to build a generator which is able to translate code of a DSL to a corresponding
implementation in an existing coding language leaving the user with an executable application.

\bigskip

EmbeddedMontiArch is a DSL for easily creating component and connector (C\&C) architectures.
C\&C Architectures consist of many componets responsible for different tasks and calculations. The
single components have clearly defined inputs of other components and clearly defined outputs to
other components via the given connectors. These architecture is designed for the purpose of being
easily testable and allowing reusability of components making it
expecially attractive to security-relevant cyber-physical systems. A possible application of C\&C
architectures is software for selfdriving cars.

\bigskip

Deep Learning has recently seen ever growing influence in many different applications and research
fields. Thus it is an important challenge being able to build deep learning systems that are easy to understand, 
easy to test and easy to reuse for effective software engineering. EMADL tries to face this challenge by extending
EmbeddedMontiArch with tools to implement deep learning algorithms. Hereby the focus is on being
independent of an actual deep learning library thus offering many different backends for the actual
implementation of the algorithms. In the currenty version of EMADL feed forward networks,
convolutional neural nets, recurrent neural nets and reinforcement learning algorithms are already
available.

\bigskip

GANs have a special architecture for generating artifical data similar to a given data-distribution.
The architecture consists of one neural net for generating the artificial data and one neural net
trying to discriminate the generated data and the data of the real data-distribution. The two
networks play a minmax game both being trained to fool the other neural net. The idea is that
eventually the generator learns to generate data similar to the data-distribution making it
impossible for the discriminator to tell apart the faked data and the real data. The challenge of
implementing GANs to EmbeddedMontiArchDL is that two independent networks need to be trained
together. To face this challenge we will try to make use of the concepts already established in the
reinforcement learning part of EMADL.

\smallskip

After EMADL has been extended by GANs succesfully, we want to evaluate our result by implementing
some toy-examples like a mnist generator for handwritten digits. Ultimatly we want to
implement the given GAN in [1] for improved road marking recognition in unconstrained datasets and
for data augmentation with GANs.

\bigskip

The following papers will be usefull on our great quest of achieving a bachelor-degree

\bibliography{research}

\begin{thebibliography}{9}

\item[] The following two papers are about possible applications of GANs in Cyber-Physical systems:

	\bibitem{Andrew}
	@inproceedings{lee2019unconstrained,
	title={Unconstrained Road Marking Recognition with Generative Adversarial Networks},
	author={Lee, Younkwan and Lee, Juhyun and Hong, Yoojin and Ko, YeongMin and Jeon, Moongu},
	booktitle={2019 IEEE Intelligent Vehicles Symposium (IV)},
	pages={1414--1419},
	year={2019},
	organization={IEEE}
	}

	Younkwan Lee, Juhyun Lee, Yoojin Hong, YeongMin Ko, Moongu Jeon,
	\textit{Unconstrained Road Marking Recognition with Generative Adversarial
	Networks},
	Machine Learning and Vision Laboratory, GIST, Gwangju, South Korea,
	\bibitem{Masseé}
	Miankuan Zhu, Lei Han, Fujian Liang, Chaoxing Xi, Lei Wu, and Zutao Zhang
	\textit{A Novel Vehicle Open Door Safety System Based on Cyclist Detection
	Using Fisheye Camera and Improved Deep Convolutional
	Generative Adversarial Nets},


	\item[] The following three papers are about the theoretic foundations and motivations behind GANs:

	\bibitem{Wijshoff, H.A.G., van Leeuwen, J.}
	Paris, France. June 9-12, 2019
	Ian J. Goodfellow,  Jean Pouget-Abadie, Mehdi Mirza, Bing Xu, David Warde-Farley,Sherjil
	Ozair†, Aaron Courville, Yoshua Bengio
	\textit{Generative Adversarial Nets}
	Departement d’informatique et de recherche operationnelleUniversit ́e de Montreal
	\bibitem{Beauqier}
	Xi Chen, Yan Duan, Rein Houthooft, John Schulman, Ilya Sutskever, Pieter Abbeel
	\textit{InfoGAN: Interpretable Representation Learning byInformation Maximizing Generative
		Adversarial Nets},
	arXiv:1606.03657v1  [cs.LG]  12 Jun 2016
	\bibitem{Biology}
	Alec Radford, Luke Metz, Soumith Chintala
	\textit{UNSUPERVISED REPRESENTATION LEARNING WITH DEEP CONVOLUTIONAL GENERATIVE ADVERSARIAL NETWORKS},
	indico Research Boston, MA, Facebook AI ResearchNew York, NY

\item[] The following two papers are about EmbeddedMontiArch and EMADL having strong impact to our
	work:

	\bibitem{Nicola}
	Gatto, Nicola
	\textit{Modeling and Training of Deep Reinforcement Learning
	based Architectures for Cyber-Physical Systems},
	RWTH Aachen University Software Engineering Group
	\bibitem{timmermann}
	Timmermanns, Thomas
	\textit{Modelling Languages for Deep Learning based
		Cyber-Physical Systems},
	RWTH Aachen University Software Engineering Group
\end{thebibliography}

\end{document}

