We are a group of three people and our seminar paper - as mentioned in the abstract - contains a case study about the EmbeddedMontiArc system. The people in this group have different preliminary knowledges which makes the writing of a case study somewhat more complicated on the one hand, and on the other hand it makes it more exciting as everyone brings up their own knowledge and idea. This seminar thesis will be based on the research questions which are mentioned in the abstract. This introductory chapter will give you an idea of our approach and methods with regard to the research questions.
\\ \\
The first research question \textbf{RQ1} was about the question "Is EmbeddedMontiArc suitable for other systems?". In order to answer this question we thought about four items namely Objective, Theory, Method and Evaluation. Concerning Objective we can say that we should implement a model in other systems such that we can see whether EmbeddedMontiArc is suitable for other systems. Apart from that, we thought about the theory of the first research question. That is, it seems to be possible since other software (e.g. autopilot, SuperMario, PacMan etc.) runs on EmbeddedMontiArc. Our method for \textbf{RQ1} will be to implement some features for PacMan or SuperMario. With reference to the evaluation of \textbf{RQ1} we will answer a questionnaire which contains following items:
\begin{itemize}
	\item Performance
	\item The effort of installing
	\item How good os the IDE integrated?
	\item Intuitiveness
\end{itemize}
The second research question \textbf{RQ2} was "Is it possible to integrate other simulators in a recent amount of work?".Here, we are geared to the same items (Objetive, Theory, Method and Evauation) as above and these items also hold for the following research questions. The objective of \textbf{RQ2} is to implement a PacMan Simulator/SuperMario Simulator. Its theory is the same like for the \textbf{RQ1} namely it should be possible as other programs are run by EmbeddedMontiArc. The method of \textbf{RQ2} is whether people with different expertises are capable of implementing (Expert vs. Non-Expert). This also needs the question "how much time it is needed?" and "How many explanations are needed in order to be able to do the implementation?". Here, the evaluation is also a questionnaire as follows:
\begin{itemize}
	\item Time to implement
	\item Help with implementation
	\item Are there any bugs?
\end{itemize}
The third research question \textbf{RQ3} deals with the question "What kind of background knowledge is needed to model C\&C in EMA?". We subdivide this question into two subquestions. The first subquestion concerns with a simple model. The corresponding objective in the first subquestion is to implement a simple model for PacMan (e.g. PacMan runs away from the ghosts or PacMan runs along the wall). The theory is "What are components and or for what are they used?". The method is more or less the same in \textbf{RQ2} ("Are people with different expertises are capable of implementing?" and "How much time and how many explanations are needed?") and "Do the people need a workshop?". The questionnaire of the evaluation is as follows:
\begin{itemize}
	\item Time to implement
	\item Help with implementation
	\item Which preliminary knowledges helped us?
	\item Quality of the components
\end{itemize}
The second subquestion of \textbf{RQ3} which refers to a more advanced model has the objective to implement an advanced model for PacMan with a good controller. The theory of the second subquestion is also "What are components and or for what are they used?". The questionnaire for the second subquestion is as follows:
\begin{itemize}
	\item Time to implement
	\item Help with implementation
	\item Which preliminary knowledges helped us?
	\item Quality of the components
\end{itemize}
Concerning the fourth research question \textbf{RQ4} we can say that the objective is the idea of improvement. There is no theory in \textbf{RQ4}. The coresponding method is to make notes during implementation and questionnaire in the method part. The questionnaire is composed as follows:
\begin{itemize}
	\item Intuitiveness
	\item Completeness of features
	\item Which bugs have been occurred?
	\item Which features have been good?
	\item Which features have been bad?
	\item Which features have been missing and how were they translated?
\end{itemize}