In the section the sampled data is analyzed. First the results of the quantitative analysis are presented, followed by statements made by the subjects and a final  summary.

\subsection{Quantitative Analysis}

To be added after implementation completion...


\subsection{Qualitative Analysis}

\subsubsection{Subject writeups}



In this section I (Sezer) will present some evaluations concerning the facts and problems of the simulator and IDE. Apart from that, I will also clarify my problems I faced during our approach.
\\
First of all, I had to understand what "C\&C" means and I also had to clarify what components are in our context. Furthermore, I had to get accustomed to the syntax of the IDE which is quite easy to handle. Moreover, the IDE offers a good overview on the used components and how components are connected with each other. Personally, I also had to get to know what a "wrapper" and a "controller" are. For someone like me who is relocated in web development, it was difficult to get into the project. Besides there are other more intuitive datatypes in EmbeddedMontiArc Studio. These are for example \textbf{B} (Boolean), \textbf{Z} (Integer), \textbf{Q} (Rational number) and \textbf{N} (Natural number). It is also possible to assign  a domain or an interval to the above-mentioned datatypes which is not always possible with other programming languages. You can also "define" units which makes it easier to comprehend your datatypes and which is not always usual for other programming languages. Apart from this, we can also create and handle with matrices in an easier way. The creation of matrices in the EmbeddedMontiArc studio is more intuitive. In my opinion, Michael's and Armin's video are good for beginners since the interaction with the components and how to program them are explained in a good manner. Besides, the relation between the single "MontiArc" programming languages are described in the tutorial videos. What I liked in our project is that we could use techniques from the software engineering domain (e.g. representation of the components) but nevertheless we faced some problems which I will mention below. What I also liked was that it is shown in the IDE whether the used packages are correct or not. What I didn't like was that there is no possibility of debugging as it is possible in other IDEs. It is very difficult to find a semantical error and the only stuff that was available is the start console of the IDE where errors are shown. But there are no methods for debugging like variable watcher or breakpoints which made the debugging for us very difficult. Honestly, I had problems to get an idea how to start with the implementation of the PacMan-game. Therefore, I had a look on my partners' code examples in order to get an inspiration. 

\subsubsection{Summary of writeups}