The following section\footnote{Author: Haller, } consists of three parts. First a brief introduction to C\&C models. The tools used for this study follow up second. Lastly, the used case study method is presented.


\subsection{C \& C models}

In the following a short introduction in Connector and Component (C\&C) model based software development is given. C\&C modeling divides a task into Components and Connectors. 

A \emph{Component} represents a computation. It has predefined inputs and outputs, where the output data is obtained by some kind of mathematical transformation of the input data. The number of inputs and outputs is arbitrary.[check?]

A \emph{Connector} represents the data flow by connecting outputs with inputs.  

By making this division, the paradigm ensures modularity and therefore reusability[ref].

Examples of this approach are Simulink[ref] and Labview[ref] which are used in the automotive domain to model behaviour of Electronic Control Units (ECUs), [add]

-> simulink, labview etc

\subsection{MontiCore and EmbeddedMontiArc}

Developed by ...

Aim:
-Useability for Domain Experts
-Ease of incorporation
-Automotive domain
-Code Generation



\subsection{Performing a case study in Software Engineering}

This study roughly follows the guidelines stated by [ref: Guidelines Case Study.pdf] by doing \emph{this \& that}.

-Quantitative Data
-Qualitative Data

Analysis?
