\subsection{IDE integration}
\begin{figure}
	\label{fig:idePacmanTop}
	\centering
	\includegraphics[scale=0.55]{pictures/IDE-PacMan-Top.pdf}
	\caption{Main options for the PacMan project in the ide}
\end{figure}
To integrate a simulator into the EmbeddedMontiArcStudio several steps were necessary. In figure. \ref{fig:idePacmanTop} you can see the top view of the EmbeddedMontiArc's ide. The five added features here are as follows:
\begin{itemize}
	\item[1.] This opens a new tab where you can play a normal game of PacMan
	\item[2.] Generate the WebAssembly of the main component
	\item[3.] This opens a new tab in which the simulation of the component takes place
	\item[4.] Generates the visualization of the main component and shows it in a new tab
	\item[5.] Generates the reporting of all components and shows it in a new tab
\end{itemize}
The features needed to be implemented properly in different places in order to work along the logic of the ide. Each one calls a batch script which again runs the jar for the demanded task for the specific files. In addition, for feature 1 and 2 extra plugins were required which got implemented by group A (expert) and can be reused for SuperMario.
A full list of the files edited can be found in the attachment.